\documentclass{beamer}
\usetheme{Pittsburgh} %White plain with round dots for bullets
\setbeamercolor{background}{fg=yellow!80!black,bg=blue!20!white}
\usecolortheme {seagull}
\useinnertheme {rectangles}
\useoutertheme{sidebar}
\usecolortheme {wolverine}

\usepackage{textcomp}
\usepackage{url}
\usepackage{pgf}
\usepackage{tikz}
\usepackage{hyperref}
\usepackage{graphics}
%\usepackage{media9}
\usepackage[english]{babel}
\title{Walking Speed and Mortality After Stroke}
\author{J Riad \& A Roberts}
\institute{Skovde Hospital, Sweden \& RJAH, UK}
\usepackage{Sweave}
\begin{document}
\input{Stroke-Beamer-concordance}
\section*{Outline}
\begin{frame}
\titlepage
\end{frame}
\section{Stroke}
\begin{frame}\frametitle{Stroke}
\begin{itemize}
\item 1 million sufferers in UK
\item 40\% have spasticity affecting function
\item Walking very often impaired
\item But not referred for treatment!
\end{itemize}
\end{frame}
\begin{frame}\frametitle{Key Problems}
\begin{itemize}
\item Spasticity - Stiff knee gait
\item Equinus - Poor clearance and instability
\item Equinus - Knee hyperextension
\item Foot varus - Instability
\item And often pain.
\end{itemize}
\end{frame}
%%%
% \begin{frame}
%   \frametitle{video}
% \begin{center}
% \href{run:foot_film.runfile.sh}{\includegraphics[scale=0.25]{Rasch-FAQ.png}}
% \end{center}
% \end{frame}

\section{So What?}
\begin{frame}\frametitle{Does Walking Matter?}
\begin{itemize}
\item ICF Perspective
\begin{itemize}
\item Structure
\item Function
\item Participation
\end{itemize}
\end{itemize}
\end{frame}
\begin{frame}\frametitle{Does Walking Matter?}
\begin{itemize}
\item Maintaining fitness
\item Survival related to walking speed
\item Stanaway et al 2011
\item Schrack et al 2010
\end{itemize}
\end{frame}
\section{Swedish Data}
\begin{frame}\frametitle{Does Walking Speed After Stroke Affect Mortality?}
	\begin{columns}
		\begin{column}{5cm}
		\begin{itemize}
		\item Swedish data
		\item 5M walking speed
		\item 10 year survival identified
		\end{itemize}
		\end{column}
		\begin{column}{5cm}
    \includegraphics[width=5cm]{Lancet_paper-flowchart.pdf}
		\end{column}
	\end{columns}
\end{frame}
\begin{frame}\frametitle{Assessment (n=141)}
	\begin{columns}
		\begin{column}{5cm}
		\begin{itemize}
		\item Visited by physiotherapist.
		\item Walking speed barefoot and in shoes
		\item ADL / Barthel categorisation
		\end{itemize}
		\end{column}
		\begin{column}{5cm}
    \includegraphics[width=5cm]{Lancet_paper-Stroke-to-exam.pdf}
		\end{column}
	\end{columns}

\end{frame}
\begin{frame}\frametitle{Follow Up at 10 Years (n=138)}
			\begin{itemize}
			\item Survival status and date of death
			\item Three subjects untraceable
			\end{itemize}
% latex table generated in R 3.0.2 by xtable 1.7-1 package
% Sun Dec 29 20:09:52 2013
\begin{table}[ht]
\centering
\caption{Mean (SD) of Age} 
\begin{tabular}{llll}
  \hline
  \hline
 & Sex &  &  \\ 
  Status at 10 Years &  & Alive & Dead \\ 
   & Male & 63.5 (14.3) & 78.4 (7.30) \\ 
   & Female & 66.6 (9.23) & 77.3 (7.36) \\ 
   \hline
\end{tabular}
\end{table}\end{frame}
\begin{frame}\frametitle{Speed and Mortality}
% latex table generated in R 3.0.2 by xtable 1.7-1 package
% Sun Dec 29 20:09:52 2013
\begin{table}[ht]
\centering
\caption{Mean (SD) of Speed in Shoes} 
\begin{tabular}{llll}
  \hline
  \hline
 & Sex &  &  \\ 
  Status at 10 Years &  & Alive & Dead \\ 
   & Male & 0.906 (0.320) & 0.698 (0.305) \\ 
   & Female & 1.08 (0.330) & 0.832 (0.320) \\ 
   \hline
\end{tabular}
\end{table}\end{frame}
\begin{frame}\frametitle{Does it Really Make a Difference?}
\begin{itemize}
  \item Take the fastest and slowest quartiles
  \item Plot their survival curves
  \item Use a Cox Proportional Hazard analysis
  \item Assess significance (Wald Statistic)
\end{itemize}
\end{frame}

\begin{frame}\frametitle{Cox Proportional Hazard Analysis}
\includegraphics[height=8cm]{Survival-Speed.pdf}
\end{frame}
\section{Random Forest Survival Analysis}
\begin{frame}\frametitle{What is Random Forest Survival Analysis?}
\begin{itemize}
  \item Classification technique taking 37\% of data as test set
  \item Classification repeated many times
  \item Best classification tree used
  \item \emph{V}ariable \emph{IMP}ortance or maximum subtree selection of variables to use
  \item Works well for high dimensional data sets
\end{itemize}
\end{frame}
\begin{frame}\frametitle{Classification as a Tree}
  \includegraphics[width=1\textwidth]{tree.png}
\end{frame}
\begin{frame}\frametitle{Age as a Classifier}
\includegraphics{Stroke-Beamer-agevar}
\end{frame}
\begin{frame}\frametitle{Sex as a Classifier}
\includegraphics{Stroke-Beamer-sexvar}
\end{frame}
\begin{frame}\frametitle{Speed Barefoot as a Classifier}
\includegraphics{Stroke-Beamer-speedbfvar}
\end{frame}
\begin{frame}\frametitle{Speed in Shoes as a Classifier}
\includegraphics{Stroke-Beamer-speedshvar}
\end{frame}
\section{So What?}
\begin{frame}\frametitle{Bad Luck or An Opportunity}
\begin{itemize}
  \item Current thinking generally sees risk of death as unalterable
  \item Epidemiological papers show risk not opportunities
  \item A working model needed for Stroke-Senility-Mortality
\end{itemize}
\includegraphics{reaper.png}
\end{frame}
\begin{frame}\frametitle{Entering Death's Dominion}
\includegraphics{Speed-Stroke-Model.pdf}
\end{frame}
\begin{frame}\frametitle{Avoiding Death's Dominion}
\includegraphics{Speed-Stroke-Rx-Model.pdf}
\end{frame}
\begin{frame}\frametitle{Mapping Benefit from Improving Speed}
\includegraphics[width=1\textwidth]{Risk-Reduction.png}
\end{frame}
\section{Conclusion}
\begin{frame}\frametitle{Conclusion}
\begin{itemize}
  \item Slow walking after stroke predicts higher mortality
  \item Improving walking speed \emph{might} reduce mortality
  \item Further work needed....
\end{itemize}
\end{frame}




\end{document}
